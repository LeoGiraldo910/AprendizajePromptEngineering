\documentclass{beamer}
\usetheme{Madrid}
\usecolortheme{seagull}
\usefonttheme{professionalfonts}

\title{Inteligencia Artificial en Mantenimiento Predictivo}
\subtitle{Predicción de Fallas en Equipos Electrónicos y Sistemas}
\author{Cristian Leonardo Giraldo Pabón}
\date{\today}

\begin{document}

\begin{frame}
  \titlepage
\end{frame}

\begin{frame}{Contenido}
  \tableofcontents
\end{frame}

\section{Introducción}
\begin{frame}{Introducción}
  \begin{itemize}
    \item El mantenimiento predictivo es esencial para prevenir costosas interrupciones en la operación de equipos electrónicos y sistemas.
    \item En esta presentación, exploraremos cómo la inteligencia artificial puede mejorar la predicción de fallas.
    \item Analizaremos su importancia y sus aplicaciones en la industria.
  \end{itemize}
\end{frame}

\section{Inteligencia Artificial en Mantenimiento Predictivo}
\begin{frame}{Inteligencia Artificial en Mantenimiento Predictivo}
  \begin{itemize}
    \item La inteligencia artificial (IA) juega un papel crucial en el mantenimiento predictivo.
    \item Tipos de algoritmos de IA utilizados: redes neuronales, máquinas de soporte vectorial, etc.
    \item Recopilación de datos: sensores, IoT y sistemas de monitoreo en tiempo real.
    \item Procesamiento y análisis de datos: técnicas de aprendizaje automático, incluyendo el aprendizaje profundo.
  \end{itemize}
\end{frame}

\section{Aplicaciones Prácticas}
\begin{frame}{Aplicaciones Prácticas}
  \begin{itemize}
    \item Caso de estudio 1: Mantenimiento predictivo en aerogeneradores para maximizar la eficiencia y reducir costos de reparación.
    \item Caso de estudio 2: Predicción de fallas en sistemas de refrigeración en centros de datos.
    \item Ventajas y beneficios de la predicción de fallas utilizando IA en términos de costos, tiempo y seguridad.
  \end{itemize}
\end{frame}

\section{Desafíos y Futuro}
\begin{frame}{Desafíos y Futuro}
  \begin{itemize}
 \item Desafíos en la implementación de IA en mantenimiento predictivo: datos de calidad, escalabilidad, interpretabilidad de modelos, etc.
    \item Futuras tendencias y avances: IA en sistemas de vehículos autónomos, robótica industrial, y más.
    \item La IA está en constante evolución, y su papel en el mantenimiento predictivo seguirá creciendo.
  \end{itemize}
\end{frame}

\section{Conclusiones}
\begin{frame}{Conclusiones}
  \begin{itemize}
    \item La inteligencia artificial es una herramienta poderosa en la predicción de fallas y el mantenimiento predictivo.
    \item Su aplicación puede mejorar la eficiencia operativa y reducir los costos de mantenimiento.
    \item Es esencial estar al tanto de las tendencias y desafíos en este campo en constante evolución.
  \end{itemize}
\end{frame}


\end{document}
